\documentclass{article}
\usepackage{listings}
\usepackage{graphicx}

\lstset{ %
language=Java,                  % the language of the code
basicstyle=\footnotesize,       % the size of the fonts that are used for the code
numbers=left,                   % where to put the line-numbers
numberstyle=\footnotesize,      % the size of the fonts that are used for the line-numbers
stepnumber=1,                   % the step between two line-numbers. If it's 1, each line 
                                % will be numbered
numbersep=5pt,                  % how far the line-numbers are from the code
showspaces=false,               % show spaces adding particular underscores
showstringspaces=false,         % underline spaces within strings
showtabs=false,                 % show tabs within strings adding particular underscores
frame=single,                   % adds a frame around the code
tabsize=2,                      % sets default tabsize to 2 spaces
captionpos=t,                   % sets the caption-position to bottom
breaklines=true,                % sets automatic line breaking
breakatwhitespace=false,        % sets if automatic breaks should only happen at whitespace
title=\lstname,                 % show the filename of files included with \lstinputlisting;
}

\usepackage[parfill]{parskip}
\begin{document}

\title{Forward and backward-chaining for Testing Satisfiability of Propositional Horn Clauses \protect\\ Midterm Report}
\author{Luke Slater}

\maketitle

\section{Introduction}

The problem I propose to investigate is one of testing entailment and
satisfiability of given statements together with a knowledgebase, for a particular 
type of logical structure. Issues of 
logical knowledge representation and the possibility of automatically reasoning
with them have been considered for a long time in the fields of Computer Science
and Mathematics, and many unsolved problems
remain in the field. Even for first-order logic it is not possible to create a
decidable algorithm to prove that any given statement is entailed.

As such, this work will use one particular subset of first-order logic
represented in Horn clauses, which allows for relatively easy and
decidable testing of entailment. The two approaches considered seem
superficially quite similar, but are actually vastly different in terms of
time complexity and decidability.

\section{Description of Problem}

\subsection{Horn Clauses}

\subsection{Satisfiability}

The process of resolution when concerned with propositional Horn clauses is
called refutation-complete\cite{refcomp}, which means that if there is a
contradiction in a set of clauses, there is a path to that contradiction through
the process of resolution.\cite{resolution}

This means, on the basis of the inverse satisfiability problem\cite{invsat},
that resolution can be used to determine the satisfiability of a
literal given a knowledgebase - rather, whether the knowledgebase entails the
given literal.

This is possible because the assertion KB $\vdash$ T is logically equivalent to
KB $\cup$ $\neg$T $\vdash$ $\bot$. That is, one can prove that the knowledgebase
entails a given input literal by deriving a contradiction from the negation of
the literal with the knowledgebase.

However, just because there is a potential route to the contradiction possible,
doesn't necessarily mean that it will be found by a particular implementation of
resolution. In terms of implementation there are two primary methods, 
forward-chaining and backward-chaining.





The investigation will be into two methods of deciding whether a given
knowledgebase entails a set of literals in an input. Both the knowledgebase and
the input are structured in the form of Horn clauses.

Horn clauses are a method of representing simple conditional 'if; then'
sentences, sometimes called 'rules,' which form a subset of first-order logic 
for which satisfiability is actually decidable through automated reasoning.

The knowledgebase is constructed with a set of positive Horn clauses, each of
which includes a set of negative literals and exactly one positive literal,
which essentially means that the set of negative literals entail the positive
literal.

The input, on the other hand, is a negative Horn clause containing only negative
literals, which is sometimes called a 'Goal clause,' and intuitively means 
'show that all these hold.'

Both algorithms presented offer a method of deciding whether the knowledgebase
entails a given negative Horn clause, and do this by 
attempting to produce a derivation of the empty negative clause from the
knowledgebase together with the negative Horn clause input.

This investigation will be limited to the propositional case (variable-free), 
as in the first-order case neither solution is decidable. Rather, there is no 
decidable solution to determining whether first-order Horn clauses entail
anything.

\section{Algorithms}

All algorithms operate on an input of a set of positive horn Clauses

\subsection{Forward Chaining}

Forward chaining instead works by starting with the facts in the knowledgebase
and works towards the goals given as input. It marks each atomic constituent of
the input as 'solved' once it finds they are entailed by the knowledgebase.

There are many variations of the forward-chaining approach to Horn clause
satisfiability, but the one considered here will be a polynomial-time approach.
The naive implementation simply 

\subsubsection{Description}

\subsubsection{Correctness}

\subsubsection{Time Complexity}

\subsection{Backward Chaining}

Backward chaining starts with a list of goals as input, which correspond to the
literals in the negative clause provided as input we are trying to prove.

The algorithm begins by selecting the first goal in the list, and looks for a
clause in the knowledgebase whose positive literal is equal to it. It then
calls itself recursively with the negative literals from the correspondent clause
in the knowledgebase as 'sub-goals,' followed by the rest of the input goals.

It continues to call itself recursively until reaching the end of the list, in
which case it returns 'true.' If the sub-goals cannot be solved, then other
clauses in the knowledgebase with the corresponding positive literal are
considered. If one cannot be found, the algorithm returns false. A goal is
solved when it leads to a positive clause in the knowledgebase with no negative
literals.

The algorithm is called backward chaining because it works backwards, from the
goals to facts in the database. It is also depth-first, since it solves the
sub-goals turned up by each input before progressing to the next item in the
input.

It is considered to be rather inefficient, as it will generally end up
performing many redundant searches, which can lead to exponential time
complexity in large knowledgebases. Furthermore, it can also end up looping
infinitely for certain knowledgebases.

\subsubsection{Description}

\subsubsection{Correctness}

This algorithm will always return the correct answer only if it finishes.

\subsubsection{Time Complexity}

The worst-case for this algorithm will always be infinite, since the algorithm
is undecidable. However, while the algorithm is undecidable, in most cases it
will return the correct answer, and depending on the input conditions the 

\subsection{Discussion}


\begin{thebibliography}{9}

\bibitem{krr}
  Ronald J. Brachman, Hector J. Levesque.
  \emph{Knowledge Representation \& Reasoning}.
  2004.

\bibitem{horn}
  Alfred Horn.
  \emph{On sentences which are true of direct unions of algebras}
  Journal of Symbolic Logic, 16:14-21
  1951.
\end{thebibliography}

\end{document}
